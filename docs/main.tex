\documentclass{article}
\usepackage{amsmath}

\begin{document}
\begin{center}
{\Large{}Development of RANS boundary condition for
  AMR-Wind}{\Large\par} \par\end{center}

Document all notes here.

\section{RANS ABL grid stretching}

The physical domain coordinates $x$ and the computational domain $\xi$
are related by the mapping function $x(\xi)$:

$$
x(\xi) = x_{geom}(\xi) (1-h(\xi)) + h(\xi) x_{const}(\xi)
$$
where the blend function $h(\xi)$ is defined as  
$$
h(\xi, \xi_T, W) = \frac{1}{2} \left( 1 + \tanh \left( \frac{\xi-\xi_T}{W} \right) \right)
$$
and depends on the transition location $\xi_T$ and width W.  
$$
\frac{d x}{d \xi} = (1-h) x'_{geom} - h' x_{geom} + h x'_{const} + h' x_{const}
$$

\subsection{Geometry growth mapping function}

The geometric growth function $x_{geom}$ is defined as :  
$$
x_{geom}(\xi) = \frac{\Delta_0}{r}  \frac{1-r^{\xi+1}}{1-r} - \frac{\Delta_0}{r}
$$

With the first cell height $\Delta_0$ and stretch rate $r>1$ provided
by the user.

The inverse of this function is
$$
\xi_{inv}(x) = \frac{\ln \left( 1- (x-\Delta_0/r) \frac{r (1-r)}{\Delta_0} \right)}
{\ln (r)}
$$

The derivative of this function is:
$$
\frac{d x_{geom}}{d\xi} = \frac{\Delta_0}{r (r-1)} r^{\xi-1} \ln(r)
$$

\subsection{Constant cell size mapping function}

$$
x_{const} = s(\xi-\xi_0) + L_{match}
$$

slope $s$, offset $\xi_0$, and match height $L_{match}$.

To set up the constants in $x_{const}$, there are two options:

1.  We can compute the $\xi$ that would correspond to $L$ input by the user:

$$
\xi_L = \textrm{int}\{ x_{inv}(L, \delta_0, r) \}
$$

2.  We can compute the $\xi$ that would correspond to $\Delta_{max}$ input by the user:

$$
\xi_L = \textrm{floor} \left[ \frac{\ln (\Delta_{max} r^2/\Delta_0)}{\ln r} \right] - 1
$$

We'll set the offset $\xi_0$=$\xi_L$. Because $\xi_L$ has been rounded to the nearest integer, we'll compute $L_{match}$ at $\xi_L$

$$
L_{match} = x_{geom}(\xi_L, \delta_0, r)
$$

And the slope will be the cell size at $\xi_L$: 

$$
s =  x_{geom}(\xi_L, \delta_0, r) -  x_{geom}(\xi_L-1, \delta_0, r)
$$

The derivative of $x_{const}$ is

$$
\frac{d x_{const}}{d\xi} = s
$$

\section{RANS k-$\epsilon$ ABL boundary conditions}

Implemented in \texttt{MOData.cpp} as \texttt{calc\_phi\_m\_alinot()}: 
\begin{equation}
\phi_m\left(\zeta \right) =
\begin{cases}
\left( 1 - \beta_m \zeta \right)^{-1/4}, & \zeta < 0\\
1+ \gamma_m \zeta                       & \zeta > 0\\
\end{cases}
\end{equation}

where $\zeta=z/L$ (see \cite{alinot2005k})

Implemented in \texttt{MOData.cpp} as \texttt{calc\_phi\_eps\_alinot()}: 
\begin{equation}
  \phi_\epsilon\left( \zeta \right) =
  \begin{cases}
    1-\zeta                            & \zeta < 0\\
    \phi_m\left( \zeta \right) - \zeta & \zeta > 0
  \end{cases}
  \end{equation}

The following calculations are implemented in \texttt{ShearStress.H}.

As \texttt{ShearStressAlinot.calc\_mu()}
\begin{equation}
  \mu_{t0}(z) = \frac{\rho \kappa u_* z}{\phi_m(\zeta)}
\end{equation}

As \texttt{ShearStressAlinot.calc\_eps()}
\begin{equation}
  \epsilon_0(z) = \frac{u_*^3}{\kappa z} \phi_\epsilon(\zeta)
\end{equation}

As \texttt{ShearStressAlinot.calc\_tke()}
\begin{equation}
  k_0(z) = \sqrt{\frac{\kappa u_* z \epsilon_0(z)}{C_\mu \phi_m}}
\end{equation}

\begin{equation}
  k_0(z) = \sqrt{\frac{\mu_{t0}(z)\epsilon_0(z)}{\rho C_\mu}}
\end{equation}


As \texttt{ShearStressAlinot.calc\_omega()}
\begin{equation}
  \omega_0(z) = \frac{\epsilon_0(z)}{C_\mu k_0(z)}
\end{equation}

\bibliographystyle{acm}
\bibliography{references}
\end{document}
